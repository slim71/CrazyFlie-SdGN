\chapter*{Fasi di Test}
\addcontentsline{toc}{chapter}{Fasi di Test}

Detto quanto basta per poter comprendere il codice che siamo andati a scrivere, elenchiamo ora le varie fasi di test che abbiamo attraversato per arrivare all’esperimento finale. Le fasi sono tra loro progressive. Per alcuni test è disponibile sia il relativo file omonimo (.py), sia un video omonimo (.mp4) in cui si mostra il risultato di ciascuna esecuzione. 
\\
\\
TEST\_1.py: 
\\
In questo file abbiamo testato le funzioni della libreria py-vicon e della cf-lib che meglio riuscissero a far si che il Drone, dopo una iniziale fase di decollo, iniziasse a inseguire la posizione della Wand. In questo caso il nuovo riferimento di posizione non era dato ad ogni istante ma ad intervalli di tempo prefissati. All’interno di ogni intervallo il riferimento era mantenuto costante e pari all’ultimo inviato. Il risultato è un \textbf{“inseguimento a tratti”} della Wand da parte del Drone. 
\\
\\
Il Drone deve essere posizionato esattamente al centro della stanza e orientato esattamente come la terna fissa “V”. Soltanto dopo essersi assicurati che sia così può essere acceso e può essere eseguito l’esperimento. 
\\
\\
\\
TEST\_2.py: 
\\
A questo punto abbiamo eliminato il vincolo relativo al riferimento costante all’interno degli intervalli di tempo ed abbiamo iniziato a inviare in ogni istante (ad una frequenza di 100 Hz) la posizione corrente della Wand. Si ha dunque un “\textbf{inseguimento in tempo reale}”. 
\\
\\
\\
TEST\_3.py: 
\\
Dato che fino a questo momento l’esperimento non aveva un vero “\textbf{criterio di arresto}”, abbiamo deciso di far interrompere l’esperimento facendo atterrare gradualmente il Drone a partire dalla posizione corrente nel momento in cui la Wand viene spenta. La logica adottata è la stessa già spiegata in precedenza per cui al momento dello spegnimento della Wand si ha l’invio da parte del Vicon della particolare posizione della Wand pari a  [0, 0, 0] che abbiamo interpretato come codice di errore e che, qualora fosse fornita per diversi istanti consecutivi da logo all’inizio della procedura di atterraggio. 
\\
\\
\\
TEST\_4.py:
\\
Qui è dove abbiamo eliminato il “vincolo” di dover iniziare l’esperimento facendo partire il Drone dall’origine del sistema Vicon e con orientazione relativa rispetto alla terna “V” nulla. 
Per fare ciò abbiamo utilizzato la posizione e orientazione del Drone rispetto alla terna Vicon per creare le relative matrici omogenee di rototraslazione che permettessero di ricevere informazioni in terna Vicon e inviare riferimenti al Drone in terna Body in modo del tutto coerente secondo quanto spiegato in precedenza.
\\
\\
Il Drone può quindi essere \textbf{inizialmente posizionato ed orientato in modo del tutto arbitrario} (purchè parta ovviamente sempre dal suolo e senza pendenze).  
\\
\\
